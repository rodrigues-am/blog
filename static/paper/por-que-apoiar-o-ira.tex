% Created 2025-06-23 seg 19:56
% Intended LaTeX compiler: lualatex
\documentclass[twocolumn]{article}
\usepackage{amsmath}
\usepackage{fontspec}
\usepackage{graphicx}
\usepackage{longtable}
\usepackage{wrapfig}
\usepackage{rotating}
\usepackage[normalem]{ulem}
\usepackage{capt-of}
\usepackage{hyperref}
\usepackage{fontspec}
\setmainfont{EB Garamond} % ou outra Garamond disponível no sistema
\setmainfont{Roboto} % Aplica a fonte Roboto a todo o documento
\setmainfont{Roboto}[UprightFont = *-Regular,BoldFont    = *-Bold, ItalicFont  = *-Italic]
\usepackage{setspace}
\linespread{1} % espaçamento entre linhas
\usepackage{parskip}
\setlength{\parskip}{0.2\baselineskip} % espaçamento entre parágrafos
\setlength{\parindent}{14pt} % remove indentação (opcional, se quiser estilo de artigo de opinião)
\usepackage{xcolor}
\definecolor{darkgray}{gray}{0.3}
\usepackage{fancyhdr}
\pagestyle{fancy}
\fancyhf{}
\fancyhead[L]{\textcolor{darkgray}{\textbf{OPINIÃO}}}
\fancyhead[R]{\textcolor{darkgray}{22 de junho de 2025}}
\renewcommand{\headrulewidth}{.5pt}
\renewcommand{\headrule}{{\color{darkgray}\hrule width\headwidth height\headrulewidth \vskip-\headrulewidth}}
\fancyheadoffset[L]{0cm}
\fancyheadoffset[R]{0cm}
\fancyhfoffset[L]{0cm}
\fancyhfoffset[R]{0cm}
\setlength{\headwidth}{\dimexpr\paperwidth-2in}
\usepackage[top=2cm,bottom=2.5cm,left=2.5cm,right=2.5cm]{geometry}
\setlength{\headwidth}{\linewidth}
\usepackage{multicol}
\setlength{\columnsep}{1cm}
\usepackage{titling}
\setlength{\droptitle}{0cm} % espaço acima do título
\preauthor{\vspace{4em}\begin{center}\large} % reduz antes do autor
\postauthor{\end{center}\vspace{0m}}       % reduz depois do autor
\predate{\vspace{0em}\begin{center}\small}
\postdate{\end{center}\vspace{0em}}
\pretitle{\begin{flushleft}\headerfont\LARGE\bfseries}
\posttitle{\end{flushleft}}
\preauthor{\begin{flushleft}\headerfont\large}
\postauthor{\end{flushleft}}
\predate{\begin{flushleft}\headerfont\small}
\postdate{\end{flushleft}}
\makeatletter
\let\oldmaketitle\maketitle
\renewcommand{\maketitle}{\oldmaketitle\thispagestyle{fancy}}
\author{André Rodrigues}
\date{}
\title{Por que apoiar o Irã?}
\hypersetup{
 pdfauthor={André Rodrigues},
 pdftitle={Por que apoiar o Irã?},
 pdfkeywords={},
 pdfsubject={},
 pdfcreator={Emacs 29.4 (Org mode 9.7.29)}, 
 pdflang={Portuges}}
\begin{document}

\maketitle
Na madrugada do dia 13 de junho, Israel lançou um ataque contra o Irã, em meio às negociações entre Teerã e Washington sobre o programa nuclear iraniano. Poucos dias depois, em 21 de junho, os Estados Unidos também atacaram centros de processamento nuclear no Irã — os mesmos alvos atingidos inicialmente por Israel.

O principal argumento para justificar esses ataques unilaterais e ilegais é que o Irã estaria direcionando seu programa nuclear ao desenvolvimento de armamentos atômicos. A ideia seria que um bombardeio às instalações enfraqueceria o programa nuclear como um todo e impediria o país de adquirir capacidade bélica nuclear.

No entanto, em março, em depoimento ao Congresso dos Estados Unidos, a diretora de inteligência norte-americana, Tulsi Gabbard, afirmou não haver qualquer evidência de que o Irã estivesse desenvolvendo armas nucleares. Ainda assim, no fim de maio, a Agência Internacional de Energia Atômica (AIEA), sediada na Áustria, publicou um relatório ambíguo, afirmando não poder garantir que o programa iraniano era exclusivamente pacífico. Esse relatório foi usado como pretexto para legitimar os ataques.
\section*{Carnificina e destruição: o mesmo roteiro de sempre}
\label{sec:org51dac40}

Não é preciso muito esforço para lembrar o que marcou as primeiras décadas do século XXI: um período iniciado pelos atentados às torres gêmeas em 2001 e pela subsequente “guerra ao terror”.

Sob o pretexto de capturar Osama bin Laden, a OTAN — sob liderança dos Estados Unidos — invadiu e ocupou o Afeganistão, uma guerra que duraria 20 anos. A coalizão dos países mais ricos do mundo ocupou um dos países mais pobres, provocando o deslocamento de milhões de pessoas e a morte de dezenas de milhares de civis. Ao longo da ocupação, multiplicaram-se os casos de violações de direitos humanos e crimes de guerra cometidos pelo Exército norte-americano.

Internamente, a chamada \emph{"Patriot Act"} suspendeu uma série de garantias civis nos Estados Unidos, criando mecanismos de vigilância e controle da população em nome da segurança nacional, enquanto as guerras se expandiam pela Ásia.

A invasão do Iraque, em 2003, expôs com ainda mais clareza o modus operandi do imperialismo. Naquele momento, acusava-se o governo de Saddam Hussein de desenvolver armas de destruição em massa — uma alegação que jamais foi comprovada. Sob essa justificativa forjada, deu-se início à ocupação militar do país. O saldo foi catastrófico: centenas de milhares de civis mortos, milhões de deslocados, crimes de guerra, e a destruição quase total do país. Mesmo com a retirada formal das tropas em 2011, os EUA mantêm até hoje bases militares no território iraquiano.

Em 2011, foi a vez da Líbia. Sob o argumento de remover Muammar Gaddafi — então acusado de desenvolver um programa nuclear — a OTAN bombardeou o país, matando milhares de civis e mergulhando o que era uma das nações mais desenvolvidas da África em um colapso social e político irreversível.

Somália, Sudão, Iêmen — os exemplos se multiplicam ao longo das duas primeiras décadas deste século. Em todos eles, o padrão se repete: o discurso da intervenção "preventiva" serve de cortina para massacres, destruição e aprofundamento da miséria.
\section*{Irã: soberania e resistência}
\label{sec:orgd328dfc}

Em 1953, os governos do Reino Unido e dos Estados Unidos orquestraram um golpe de Estado no Irã, derrubando o primeiro-ministro Mohammad Mossadegh, que havia nacionalizado o petróleo até então controlado pelos britânicos. Em seu lugar, reinstauraram a ditadura do xá Mohammad Reza Pahlavi, governo títere do imperialismo.

O regime do xá ficou marcado pela brutalidade: torturas, execuções e censura sistemática da oposição. Sua polícia secreta, a SAVAK, tornou-se sinônimo de terror, uma das mais brutais conhecidas.

Não são poucas as ditaduras impostas pelo imperialismo americano e europeu ao longo do século XX — também nós, na América Latina, conhecemos bem isso.

Foi apenas em 1979, após uma ampla mobilização popular, que o regime do xá foi derrubado na Revolução Iraniana. Desde então, o país tem enfrentado sanções econômicas severas, que penalizam principalmente a população civil. Trata-se de uma arma recorrente do imperialismo contra todos os povos que ousam desafiar sua dominação.

Aos que defendem que o Irã não deve ter acesso a armamentos nucleares, é preciso lembrar que várias nações já possuem essas armas — mas, até hoje, apenas uma as utilizou. Foram os Estados Unidos, que lançaram bombas atômicas sobre as populações civis de Hiroshima e Nagasaki, causando centenas de milhares de mortos e feridos. São esses mesmos senhores que hoje tentam convencer o mundo sobre o perigo iraniano.
\section*{A luta palestina}
\label{sec:orgf2660bb}

O Irã é hoje o único país que, de forma concreta, apoia a resistência palestina e denuncia o genocídio em curso na Faixa de Gaza. Enquanto governos árabes assistem passivamente ao massacre — e, em muitos casos, colaboram com Tel Aviv —, Teerã tem oferecido apoio político e material à causa palestina. E é justamente por isso que o país se tornou alvo de constantes ataques.

A retaliação iraniana aos bombardeios israelenses também deve ser vista no contexto da luta pela libertação da Palestina. Gaza é atacada diariamente; os mortos já se contam em centenas de milhares, e Israel segue impune. A resposta do Irã representa, portanto, uma ruptura com essa lógica de impunidade.

Diante da história, dos fatos e das consequências da dominação imperialista, considero que uma vitória da República Islâmica do Irã é efetivamente um avanço para a luta de todos os povos oprimidos e para a humanidade como um todo.

É por isso que apoio o Irã.
\end{document}
